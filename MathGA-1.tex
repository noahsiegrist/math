\documentclass{article}

\usepackage[utf8]{inputenc}
\usepackage[ngerman]{babel}
\usepackage{color}
\usepackage{amssymb}
\usepackage{amsmath}
\usepackage{amsthm}
\usepackage{datetime}
\usepackage{graphicx}
\usepackage{hyperref}

\setlength\parindent{0pt}

\title{Mathe GA-1}
\author{Jan, Mariel, Noah Siegrist}
\date{\today}

\begin{document}

\maketitle	
\tableofcontents
\newpage

\section{Logik}
Die Logik dient dazu komplexe Probleme analytisch zu Lösen. Sie bricht die Sprache auf eine abstrakte form herunter, die eindeutig und genau ist. Somit ist die Logik ein Grundbaustein aller Wissenschaften.

\subsection{Logische Aussage}

\subsubsection{Definition}

Eine (logische) Aussage kann immer als \textbf{wahr (w)} order \textbf{falsch (f)} identifiziert werden. Wenn die Aussage nicht \textbf{eindeutig} mit w oder f beantwortet werden kann, so ist die Aussage nicht logisch.

\begin{itemize}
	\item[] Eine Aussage kann auch aus mehreren \textbf{Teilaussagen} bestehen. Dann heisst sie \textbf{zusammengesetzte Aussage} oder auch \textbf{aussagenlogische Formel}.
	\item[] Die \textbf{Elementaraussage} oder auch \textbf{atomare Aussage} ist eine Aussage welche aus einer Teilaussage besteht.
\end{itemize}

\subsubsection{Beispiele}

Logische Aussagen:
\begin{itemize}
  \item Feuer gibt wärme ab. (w)
  \item Feuer kühlt. (f)
  \item Feuer ist schön. (subjektive aber bestimmbar.)
  \item Wenn das Feuer brennt, trocknen die Kleider schneller. (w)
\end{itemize}

Nicht Logische Aussagen:
\begin{itemize}
  \item Ist das Feuer warm? (Frage)
  \item Feuer kühlt und wärmt. (Widerspruch, Paradoxem)
  \item Mach Feuer! (Befehl)
  \item Das Feuer ist x grad warm. (“Aussageform” (siehe später), aber keine Aussage!)
\end{itemize}

\subsection{Aussagenlogische Verknüspfungen}
Logische Aussagen können verknüpft werden mittels Aussagenlogische Verknupfungen oder auch “Konnektoren”, “Junktoren” oder “Operatoren” gennant. Diese verknüpfungen können jeweils mit \textbf{Wahrheitstabellen} definiert werden.

\subsubsection{Konjunktion und}
A \textbf{und} B


 
\subsubsection{Disjunktion}
\subsubsection{Negation}
\subsubsection{Implikation}
\subsubsection{Aquivalenz}





\end{document}