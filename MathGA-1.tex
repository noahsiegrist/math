\documentclass[a4paper,12pt]{article}

\usepackage{blindtext} %optimizes spaces & layout stuff, can fill some blindtext so check layout behaviour%
\usepackage{avant} %font%
\usepackage{microtype} %justification of document%
\usepackage{wrapfig} %wrap text around pictures%
\usepackage{enumitem} %optimize list layout, more compact%
\usepackage{fancyhdr} %testing styles%
\usepackage{index} %custom indexing%
\usepackage[singlespacing]{setspace} %set spacing property%
\usepackage{tabularx} %format tabular%
\usepackage{makecell}
\usepackage{tikz}

\usepackage[utf8]{inputenc} %Umlaute etc.%
\usepackage[ngerman]{babel} %German charset%
\usepackage{color}
\usepackage{amssymb}
\usepackage{amsmath}
\usepackage{amsthm}
\usepackage{datetime}
\usepackage{graphicx}
\usepackage{hyperref}
\setlength{\parskip}{1em}
\setlength\parindent{0pt}

\usepackage{geometry}
\geometry{a4paper, total={170mm,257mm},left=30mm,right=20mm,top=25mm,bottom=15mm}

\makeindex

\title{\large{\textbf{Mathe GA-1}}}
\author{Mariel Alaba, Jan Meier, Noah Siegrist}
\date{\today}

\usepackage{titling}
\renewcommand\maketitlehooka{\null\mbox{}\vfill}
\renewcommand\maketitlehookd{\vfill\null}

\begin{document}

\begin{titlingpage}
\maketitle
\end{titlingpage}

\newpage
\cfoot[plain]{scrheadings}
\tableofcontents

\newpage

\pagenumbering{arabic}
\setcounter{page}{1}
\pagestyle{fancy}
\fancyhf{} %clear footers style%
\fancyhead[LE,RO]{\nouppercase{\leftmark}}
\fancyfoot[LE,RO]{\centering{\thepage}}

\newpage

\section{Logik}
\enlargethispage{\baselineskip}
Die Logik dient dazu komplexe Probleme analytisch zu Lösen. Sie bricht die Sprache auf eine abstrakte Form herunter, die eindeutig und genau ist. Somit ist die Logik ein Grundbaustein aller Wissenschaften.

\subsection{Logische Aussage}

\subsubsection{Definition}

Eine (logische) Aussage kann immer als \textbf{wahr (w)} order \textbf{falsch (f)} identifiziert werden. Wenn die Aussage nicht \textbf{eindeutig} mit w oder f beantwortet werden kann, so ist die Aussage nicht logisch.

\setlist{nolistsep}
\begin{itemize}
	\item[] Eine Aussage kann auch aus mehreren \textbf{Teilaussagen} bestehen. Dann heisst sie \textbf{zusammengesetzte Aussage} oder auch \textbf{aussagenlogische Formel}.
	\item[] Die \textbf{Elementaraussage} oder auch \textbf{atomare Aussage} ist eine Aussage welche aus einer Teilaussage besteht.
\end{itemize}

\subsubsection{Beispiele}

Logische Aussagen:
\begin{itemize}
  \item Feuer gibt wärme ab. (w)
  \item Feuer kühlt. (f)
  \item Feuer ist schön. (subjektive aber bestimmbar.)
  \item Wenn das Feuer brennt, trocknen die Kleider schneller. (w)
\end{itemize}

Keine Logische Aussagen:
\begin{itemize}
  \item Ist das Feuer warm? (Frage)
  \item Feuer kühlt und wärmt. (Widerspruch, Paradoxem)
  \item Mach Feuer! (Befehl)
  \item Das Feuer ist x grad warm. (“Aussageform” (siehe später), aber keine Aussage!)
\end{itemize}
\newpage
\subsection{Aussagenlogische Verknüpfungen}
Logische Aussagen können verknüpft werden mittels Aussagenlogische Verknüpfungen oder auch “Konnektoren”, “Junktoren” oder “Operatoren” gennant. Diese Verknüpfungen können jeweils mit \textbf{Wahrheitstabellen} definiert werden.

\subsubsection{Negation (nicht)}
Die negation ist keine Verknüpfung sondern invertiert eine Logische Aussage. Sie bindet am stärksten und bildet somit auch ein Literal. 

\textbf{Nicht} A wird wie folgt geschrieben: \( \neg A \)

\subsubsection*{Wahrheitstabelle}
\begin{tabular}{c || c}
  A & \( \neg A \) \\
  \hline
  0  & 1 \\
  1  & 0\\
\end{tabular}\break

\subsubsection*{Beispiele}

\begin{itemize}
  \item Feuer ist \textbf{nicht} kalt. (w)
  \item Feuer ist \textbf{nicht} heiss. (f)
\end{itemize}

\subsubsection{Konjunktion (und)}
A \textbf{und} B wird wie folgt geschrieben: \( A \land B \)

Sie bindet am zweit stärksten direkt nach der Negation.
\subsubsection*{Wahrheitstabelle}
\begin{tabular}{c|c || c}
  A & B & \( A \land B \) \\
  \hline
  0 & 0 & 0 \\
  0 & 1 & 0\\
  1 & 0 & 0\\
  1 & 1 & 1\\
\end{tabular}

\subsubsection*{Beispiele}

\begin{itemize}
  \item Feuer ist kalt \textbf{und} Feuer ist flüssig. (f)
  \item Feuer ist kalt \textbf{und} Feuer ist eine chemische Reaktion. (f)
  \item Feuer ist heiss \textbf{und} Feuer ist flüssig. (f)
  \item Feuer ist heiss \textbf{und} Feuer ist eine chemische Reaktion. (w)
\end{itemize}


\subsubsection{Disjunktion (oder)}
A \textbf{oder} B wird wie folgt geschrieben: \( A \lor B \)

Sie bindet ebenfalls am zweit stärksten direkt nach der Negation.
\subsubsection*{Wahrheitstabelle}
\begin{tabular}{c|c || c}
  A & B & \( A \lor B \) \\
  \hline
  0 & 0 & 0 \\
  0 & 1 & 1\\
  1 & 0 & 1\\
  1 & 1 & 1\\
\end{tabular}\break

Beim Logischem oder ist anzumerken, dass es sich vom entweder oder unterschiedet indem, dass wenn beide Seiten korrekt sind ist das Ergebnis auch Korrekt.

\subsubsection*{Beispiele}

\begin{itemize}
  \item Feuer ist kalt \textbf{oder} Feuer ist flüssig. (f)
  \item Feuer ist kalt \textbf{oder} Feuer ist eine chemische Reaktion. (w)
  \item Feuer ist heiss \textbf{oder} Feuer ist flüssig. (w)
  \item Feuer ist heiss \textbf{oder} Feuer ist eine chemische Reaktion. (w)
\end{itemize}

\subsubsection{Implikation (wenn, dann)}

\textbf{Wenn} A, \textbf{dann} B wird wie folgt geschrieben: \( A \implies B \)

Sie bindet am dritt stärksten nach der Konjunktion und der Disjunktion.
\subsubsection*{Wahrheitstabelle}
\begin{tabular}{c|c || c}
  A & B & \( A \implies B \) \\
  \hline
  0 & 0 & 1 \\
  0 & 1 & 1\\
  1 & 0 & 0\\
  1 & 1 & 1\\
\end{tabular}\break

Wenn A nicht war ist, kann B wahr oder falsch sein, die Aussage ist damit jedoch \textbf{nicht falsch}!

\pagebreak

\subsubsection*{Beispiele}
Wenn Max am Feuer sitzt, hat er warm.
\begin{itemize}
  \item Max sitzt nicht am Feuer und er hat kalt. 
  \textit{(Wenn Max nicht am Feuer sitzt, ist es nicht mein Problem wenn er kalt hat. Die Aussage ist nicht falsch, also wahr.)}
  \item Max sitzt nicht am Feuer und er hat warm. 
  \textit{(Wenn Max nicht am Feuer sitzt, kann er trotzdem warm haben. Die Aussage ebenfalls nicht falsch, also wahr.)
}  \item Max sitzt am Feuer und er hat kalt. 
  \textit{(Trotzdem das Max am Feuer sitzt hat er Kalt. Die Aussage ist nun falsch.)}
  \item Max sitzt am Feuer und er hat warm. \textit{(Wahr)}
\end{itemize}

\subsubsection{Äquivalenz (genau dann, wenn)}

A \textbf{Genau dann, wenn} B wird wie folgt geschrieben: \( A \iff B \)

Sie bindet ebenfalls am dritt stärksten nach der Konjunktion und der Disjunktion.
\subsubsection*{Wahrheitstabelle}
\begin{tabular}{c|c || c}
  A & B & \( A \iff B \) \\
  \hline
  0 & 0 & 1 \\
  0 & 1 & 0\\
  1 & 0 & 0\\
  1 & 1 & 1\\
\end{tabular}\break

\subsubsection*{Beispiel}
Genau dann, wenn Max am Feuer sitzt, hat er warm.
\begin{itemize}
  \item Max sitzt nicht am Feuer und er hat kalt. 
  \textit{(Wahr)}
  \item Max sitzt nicht am Feuer und er hat warm. 
  \textit{(Max hat auch warm ohne, dass er dazu am Feuer sitzen muss. Aussage falsch.)}
  \item Max sitzt am Feuer und er hat kalt. 
  \textit{(Trotzdem das Max am Feuer sitzt hat er Kalt. Aussage falsch.)}
  \item Max sitzt am Feuer und er hat warm. \textit{(Wahr)}
\end{itemize}
\subsection{Atomare Aussage}
Eine Atomare Aussage, ist eine Aussage ohne jeglichen Verknüpfungen oder Negationen. Also bei der Aussagelogischer Formel ( \(f = A \land \neg B \land B\) ) sind die Atomaren Aussagen ( \(A\) und \(B\) ).

\subsection{Literal}
Ein Literal, ist eine Aussage ohne jeglichen Verknüpfungen. Im Gegensatz zur Atomaren Aussage sind hier Negationen nicht ausgeschlossen. Also bei der Aussagelogischer Formel ( \(f = A \land \neg B \land B\) ) sind die Literal ( \(A\), \(B\) und \(\neg B\) ).



%\newpage
\subsection{Aussagenlogische Formel}
In der Aussagenlogischen Formel können nun komplexe Zusammenhänge erstellt werden. Wie zum Beispiel:

\begin{displaymath}
 f = (A \land B) \implies C
\end{displaymath}

\subsubsection{Syntax}
Regeln wie elementare Zeichen (Symbole) – oder auch Symbolgruppen – korrekt
zusammengesetzt werden dürfen. 

\subsubsection{Semantik}
Bedeutungslehre. Zwei aussagenlogische Formeln sind semantisch gleich oder äquivalent (symbolisch \(f \equiv g\)), wenn ihre Wahrheitstabellen identisch sind. Beachte das Symbol unterscheidet sich vom \( \iff \).

Daher, dass gewisse Ausdrücke Semantisch äquivalent sein können ergeben sich folgende Rechengesetze:

\begin{tabular}{l|cc}
  \textbf{Name} & \textbf{Gesetz} \\
  \hline
  Idempotenzgesetz & \(A \land A \equiv A  \) & \(A \lor A \equiv A  \)\\
  \hline
  Kommutativgesetz & \(A \land B \equiv B \land A \) & \(A \lor B \equiv B \lor A \) \\
  \hline
  Identitatsgesetz & \(A \land true \equiv A \)   & \(A \lor true \equiv true \) \\
  & \(A \land false \equiv false \) & \(A \lor false \equiv A \)\\
  \hline
  Assoziativgesetz & \((A \land B) \land C \equiv A \land (B \land C)\) & \((A \lor B) \lor C \equiv A \lor (B \lor C)\) \\
  \hline
  Absorptionsgesetz & \(A \land (A \lor B) \equiv A\) & \(A \lor (A \land B)\equiv A\) \\
  \hline
  Distributivitatsgesetz & \(A \land (B \lor C) \equiv (A \land B) \lor (A \land C)\) & \(A \lor (B \land C) \equiv (A \lor B) \land (A \lor C)\) \\
  \hline
  De Morgan Gesetz & \(\neg(A \land B)\equiv \neg A \land \neg B\) & \(\neg(A \lor B)\equiv \neg A \lor \neg B\) \\
  \hline
  Negationsgesetz & \(\neg\neg A \equiv A\)\\
\end{tabular}


Wobei sich das Absorptionsgesetz auf dem Idempotenzgesetz und dem Identitatsgesetz aufbaut.

\subsection{Normalformen}
Das Ziel der Normalformen ist es eine Komplexe Formel in eine semantisch äquivalente einfacher zu verstehende form zu bringen.
\subsubsection{Konjunktive Normalform (KNF)}
In der KNF ist es das Ziel, dass im obersten level der Aussagelogischen Formel nur Konjunktionen vorhanden sind. Diese halten so genante "Klauseln der konjunktiven Normalform" zusammen.
\begin{displaymath}
  f = f_1 \land f_2 \land ... \land f_n
\end{displaymath}

Die \(f_1\) bis \(f_n\) sind dabei die Klauseln, wobei eine solche Klausel aus Disjunktion und Literalen bestehen. Hier am Beispiel von der ersten Klausel:
\begin{displaymath}
  f_1 = L_{11} \lor L_{12} \lor ... \lor L_{1n}
\end{displaymath}

\subsubsection{Disjunktive Normalform (KNF)}

In der DNF ist es das Ziel, dass im obersten level der Aussagelogischen Formel nur Disjunktionen vorhanden sind. Diese halten so genante "Klauseln der disjunktiven Normalform" zusammen.
\begin{displaymath}
  f = f_1 \lor f_2 \lor ... \lor f_n
\end{displaymath}

Die \(f_1\) bis \(f_n\) sind dabei die Klauseln, wobei eine solche Klausel aus Disjunktion und Literalen bestehen. Hier am beispiel von der ersten Klausel:
\begin{displaymath}
  f_1 = L_{11} \land L_{12} \land ... \land L_{1n} 
\end{displaymath}


\subsubsection{Beispiele}
\begin{itemize}
  \item \((A \lor B) \land (C \lor \neg D) \land (\neg E \lor F)  \) (KNF)
  \item \((A \land B) \lor (C \land \neg D) \lor (\neg E \land F)  \) (DNF)
  \item \((A \lor B) \lor (C \land \neg D) \lor (\neg E \land F)  \) (Diese Formel kann weiter zur DNF gebracht werden indem die ersten Klammern entfernt werden dank dem Assoziativgesetz.)
  	
  	 \(A \lor B \lor (C \land \neg D) \lor (\neg E \land F)  \)
  \item \((A \lor \neg (C \land D)) \land (\neg E \lor F)  \) (Diese Formel kann weiter zur KNF gebracht werden mit dem dem de Morgan Gesetz und dem Distributivitatsgesetz.)
 	\((A \lor (\neg C \land \neg D)) \land (\neg E \lor F)  \)
 	
  	\( (A \lor \neg C) \land (A \lor \neg D) \land (\neg E \lor F)  \)
    
\end{itemize}

\subsection{Belegung und Modelle}
Unter einer Belegung verstehen wir eine Zuordnung der Atomaren Aussagen zu einem effektiven Wert. So zu sagen eine Zeile aus der Wahrheitstabelle anders repräsentiert.

Wenn diese Belegung nun die Aussagelogische Formel wahr macht, so wird die Belegung Modell genannt.

%TODO: Beispiel für Belegung und Model... Wie wird das geschrieben?

\subsection{Erfüllbarkeit}
Wenn eine Aussagenlogische Formel wird \textbf{erfüllbar} gennant, wenn mindestens ein Modell in der Wahrheitstabelle gefunden werden kann. Falls kein Modell gefunden werden kann, so handelt es sich um eine \textbf{Kontradiktion / Widerspruch}. \textit{Das warme Feuer ist kalt.}

Falls in der Wahrheitstabelle nur Modelle gefunden werden können, so nennen wir die Aussagenlogische Formel eine \textbf{Tautologie}. \textit{Das warme oder kalter Feuer.} 

\subsection{Prädikate}
In der Pradikatenlogik wollen wir jetzt nicht mehr Eigenschaften einzelner Objekte betrachten wie z.B. dem Feuer,
sondern die Eigenschaften einer ganzen Menge (mehr zu Mengen im nächsten Kapitel) an Objekten und machen darüber wieder Aussagen. So sind wir in der Lage effiziente Aussagen zu mehreren Objekten zu machen.

Wir haben eine Menge an Geometrischen formen und wollen herausfinden welche davon rund sind, können wir ein Prädikat aufstellen: \( P(x) := \) "x ist rund".


Unser Prädikat \( P(x) \) wird erst zur Aussage, wenn für x ein Objekt eingesetzt wird. Solange kein Objekt eingesetzt ist, kann das Prädikat auch nicht auf wahr oder falsch definiert werden und ist deshalb auch noch keine Aussage. Wenn nun ein Recheck für x eingesetzt wird, können wir feststellen, dass die Aussage \textit{"Rechteck ist rund"} falsch ist.

Ein Prädikat kann auch mehre Variablen enthalten, das ist dann ein n-stelliges-Prädikat. \( n = 0 \) wäre dann eine einfache Aussage.


\subsubsection*{Beispiel}
\begin{itemize}
  \item \(P(a) = \)  a ist durch 2 teilbar.
 
  \(P(2) \) ist wahr und \(P(3)\) falsch.
  \item \(Q(x, y, z) = \) x ist grösser als y und z.
 
  \(Q(3, 2, 2) \) ist wahr und \(Q(3, 2, 3)\) falsch.
  \item \(Z() = \) Ein Prädikat ohne variabel ist eine Aussage.
 
  \(Z() \) ist wahr.
\end{itemize}

Wie im Beispiel gerade gut zu sehen ist, wird das Prädikat zur Aussage sobald die Variablen durch effektive Werte ersetzt werden. Dies nennen wir dann eine \textbf{Belegung}
\newpage
\subsubsection{Quantoren}
Wir können nun, mithilfe der folgenden drei Quantoren, Logische Aussagen über die Prädikate machen.


Folgende Quantoren ergeben die (Prädikaten) Aussagen, welche wiederum entweder wahr oder falsch sind.
\begin{itemize}
  \item $\forall x : P(x)$ Für \textbf{alle} x ist $P(x)$ wahr.
  \item $\exists x : P(x)$ Für \textbf{mindestens} ein x ist $P(x)$ wahr.
  \item $\exists! x : P(x)$ Für \textbf{genau} ein x ist $P(x)$ wahr.
\end{itemize}

Es gibt ein paar Gesetze zu beachten:

\begin{tabular}{l|ccc}
  \textbf{Name} & \textbf{Gesetz} \\
  \hline
  Negationsgesetzte & \(\neg ( \forall x : P(x)) \) & $\iff$ & $\exists x : \neg P(x) $\\
   & \(\neg ( \forall x : \neg P(x))\) & $ \iff $ & $\exists x :  P(x)$  \\
   & \(\neg ( \exists x :  P(x))\) & $\iff$ & \(\forall x : \neg P(x)\)  \\   
   & \(\neg ( \exists x : \neg P(x))\) & $ \iff $ & \(\forall x :  P(x)\)  \\
  \hline
  Vertauschbarkeitssatze & \( \forall x,y : P(x, y)) \) & $\iff$ & $\forall y,x : P(x, y)) $\\
   & \( \exists x,y : P(x, y)) \) & $\iff$ & $\exists y,x : P(x, y)) $\\
   & \( \exists x \forall y : P(x, y)) \) & $\implies$ & $\forall x \exists y : P(x, y))  $\\  
  \end{tabular}

Diese Quantoren können nun mit den anderen Verknüpfungen kombiniert werden, wobei die Quantoren am \textbf{stärksten} binden.
\subsubsection{Beispiele}
"Nicht alle StarWars fans finden die letzte StarWars Reihe gut."

Dabei ist das Prädikat $S(x) = $ "x findet die letzte StarWars Reihe gut."
\begin{displaymath}
  \neg (\forall x : S(x))
\end{displaymath}
Lässt sich aber mit dem ersten Verneninungssatz vereinfachen zu:
\begin{displaymath}
  \exists x : \neg S(x)
\end{displaymath}


"Es gilt nicht, dass alle Menschen x, die eine Maske y tragen nicht Covid-19 bekommen."

$S(x, y) =$ "Mensch x trägt Maske y und hat Covid-19." 
\begin{displaymath}
  \neg ( \forall x \exists y  : \neg S(x, y))
\end{displaymath}
\begin{displaymath}
  \exists x \neg (\exists y  :  \neg S(x, y))
\end{displaymath}
\begin{displaymath}
  \exists x  \forall y  :  \neg (\neg S(x, y))
\end{displaymath}
\begin{displaymath}
  \exists x  \forall y  :   S(x, y)
\end{displaymath}

\newpage
\section{Mengenlehre}

\subsection{Grundbegriffe}

\subsubsection{Diskursuniversum}
Hier werden Objekte, Gegenstände oder Ideen angesprochen über welche wir sprechen wollen (z.B. mittels Logik). Dies wird uns durch die Mengenlehre geboten.

\subsubsection{Menge}
Eine Menge besteht aus Elementen (Objekten), diese haben idr eine Gemeinsamkeit und sind deswegen zu einer Menge zusammengeschlossen.

Es gibt ausserdem noch die Leere Menge:  $\emptyset$ oder \{\} . Diese enthält keine Elemente.

Ebenso zu erwähnen gilt es, sofern in Mengen das gleiche Element vorkommt gilt dies nicht mehr als Menge. Die Reihenfolge der Elemente innerhalb einer Menge ist irrelevant.

Weiter gilt noch das Gesetz der Gleichheit der Menge. Dies ist, sofern sämtliche Elemente einer Menge in der anderen vorhanden sind und umgekehrt. \\
$A = B$ :  $\Leftrightarrow$  $\forall x$ : \((x \in  A \Leftrightarrow x \in  B\))

\subsubsection{Schreibformen der Menge}
Es gibt insgesamt 3 Schreibformen für die Menge:
\begin{itemize}
  \item Aufzählende Form (Elementaufzählung)
  \item Beschreibende Form (Prädikatenschreibweise: \( P(x) \) )
  \item Venn Diagramme
\end{itemize}

Nachfolgend nun diese genauer erklärt:

\subsubsection{Aufzählende Form}
M := \{a,b,c\} ; eine Menge mit den Elementen a, b und c

\subsubsection{Prädikatenform / beschreibende Form \( P(x) \)}
Hier die Schreibweise für die Menge A aller Elemente x mit der Eigenschaft \( P(x) \) \\
$A := \{$x : \( P(x) \)\} \\

\subsubsection{Venn Diagramme}
Ausserdem gibt es noch Venn Diagramme zur graphischen Darstellung. Mehr dazu in den Beispielen.

\subsubsection{Zahlenmengen}
Es gibt verschiedene bekannte Zahlenmengen. Diese wären: \\
$\mathbb{N}$ : natürliche Zahlen \\
$\mathbb{Z}$ : ganze Zahlen \\
$\mathbb{Q}$ : rationale Zahlen (u.a. Brüche) \\
$\mathbb{R}$ : reelle Zahlen (periodisch oder nicht periodisch, u.a. Wurzeln oder Pi)

\subsection{Teilmengen}
Wir unterscheiden \textbf{echte Teilmengen} von \textbf{unechten Teilmengen}. \\
\textbf{Echte Teilmengen} werden mit $\subset$ angegeben. Hierbei ist die gemeinte Menge in der jeweils anderen enthalten (ohne \textnormal{sämtliche} Elemente der anderen Menge, dies \textnormal{wäre} dann die \textbf{unechte Menge}). Unechte Mengen (identische Mengen) werden mit $\subseteq$ angegeben.

Noch zu erwähnen ist, dass die Leere Menge $\emptyset$ teil einer jeden Teilmenge ist und die Menge selber jeweils auch ihre eigene Teilmenge ist (Menge A ist "unechte" Teilmenge von Menge A := A $A \subseteq A$).

\subsubsection{Mächtigkeit / Kardinalität}
Die Mächtigkeit wird durch die Anzahl der Elemente in einer Menge definiert. \\
M := \{a,b,c\} hätte z.B. die Kardinalität 3.

\subsubsection{Potenzmenge}
Die Potenzmenge entspricht aller möglichen Teilmengen einer Menge. \\
Sie wird mit \( P(C) \) dargestellt und auch System über Menge C genannt.

Sei C := \{1,2,4\}, dann ist \( P(C) \) := \{$\emptyset$,  \{1\}, \{2\}, \{4\}, \{1,2\}, \{1,4\}, \{2,4\}, C\}

\( S(C) \) bezeichnet ein Mengensystem über C \\
Beispiel: S := \{$\emptyset$,  \{2\}, \{4\}, \{1,2\}\}

Die Mächtigkeit der Potenzmenge wird folgendermassen angegeben: \\
|\( P(C) \)| := $2^{|C|}$ ; also 2 hoch der Anzahl Elemente in der Menge, hier: $2^{3}$ = 8, korrekt.

\subsection{Grundmenge}
Die Grundmenge $G$ versteht sich als die Menge unter allen Mengen, in welcher sich alle Mengen befinden. Diese wird oft als Viereck gezeichnet.

\textbf{Beispiel:} \\
Gewollt sind alle reellen Zahlen \{$x \in \mathbb{R}$\}, hier würde man die Grundmenge folgendermassen definieren: $G := \mathbb{R}$

\subsection{Mengenzeichen - Operationen}
Es gibt diverse Operationen, die in der Mengenlehre verwendet werden. Folgende werden wir behandeln:
\begin{itemize}
  \item Vereinigung $\cup$ - Mengenkonjunktion
  \item Durchschnitt $\cap$ - Mengendisjunktion
  \item Differenz $\setminus$
  \item Symmetrische Differenz $\Delta$
  \item Komplement $A^{c}$
  \item Kartesisches Produkt $\times$
\end{itemize}

\subsubsection{Vereinigung $\cup$ - Mengenkonjunktion}
Dies bezeichnet schlicht die Vereinigung von Mengen. \\
$A \cup B$ := \{$x \in G | x \in A \lor x \in B$\} \\
\textbf{Beispiele:} $A$ := \{x, y, z\}, $B$ := \{v, w\}, $A \cup B$ := \{v, w, x, y, z\}

$A \cup B$ := \{$x \in G | x \in A \lor x \in B$\} \\
$A$ := \{Frauen\} \\
$B$ := \{Männer\} \\
$A$ := \{Mariel\}, $B$ :=\{Jan, Noah\}, $A \cup B$ := \{Mariel, Jan, Noah\}

\subsubsection{Durchschnitt $\cap$ - Mengendisjunktion}
Hier wird auf den Durchschnitt oder die Gemeinsamkeit zweier Mengen abgezielt. \\
$A \cap B$ := \{$x \in G | x \in A \land x \in B$\} \\
\textbf{Beispiele:} \\
$A$ := \{\textcolor{red}{e}, \textcolor{red}{x}, y, z\}, $B$ := \{a, b, \textcolor{red}{e}, \textcolor{red}{x}\}, $A \cap B$ := \{\textcolor{red}{e}, \textcolor{red}{x}\}

$A \cap B$ := \{$x \in G | x \in A \land x \in B$\} \\
$A$ := \{Harry Potter, Hermine Granger, Ron Weasley, Severus Snape\} (:= Dumbledore's Armee) \\
$B$ := \{Lucius Malfoy, Bellatrix Lestrange, Severus Snape\} (:= Todesser) \\
$A \cap B$ := \{Severus Snape\}

\subsubsection{Differenz $\setminus$}
Bei der Differenz wird aus der ersten Menge das entnommen, was in der Zweiten auch vorhanden ist. \\
$A \setminus B$ := \{$x \in G | (x \in A \land x \not\in B)$\} \\
\textbf{Beispiele:}\\ 
$A$ := \{b, o, u, m, \textcolor{red}{a, t}\}, $B$ := \{\textcolor{red}{a, t}, x, y, z\} \\
$A \setminus B$ := \{b, o, u, m\}

$A \setminus B$ := \{$x \in G | (x \in A \land x \notin B)$\} \\
$A$ (:= vermeintliche Hauptstadt) := \{Zürich, Stockholm, London, Sydney\} \\
$B$ (:= offizielle Hauptstadt) := \{Canberra, Ottawa, Stockholm, London, Bern\} \\
 $A \setminus B$ = \{Zürich, Sydney\}

\subsubsection{Symmetrische Differenz $\Delta$}
Bei der symmetrischen Differenz werden die Gemeinsamkeiten beider Mengen ausgeschlossen.
$A \Delta B$ := \{$(x \in A \land x \not\in B) \land (x \in B \land x \not\in A)$\} \\
\textbf{Beispiele:} \\ 
$A$ := \{b, o, v, w, z\}, $B$ := \{v, w, z, u, m\} \\
$A \Delta B$ := \{b, o, u, m\}

$A \Delta B$ :=\{$x \in G | (x \in A \land x \notin B) \lor (x \in B \land x \notin A)$\} \\
$A$ (:= Coronagegner) := \{Marco Rima, Hans, Peter, Lisa\}, $B$ (:= Coronabefürworter) := \{Anneliese, Lisa, Robert\}, $A \Delta B$ := \{Marco Rima, Hans, Peter, Anneliese, Robert\} \\
\\
Anmerkung: Coronagegner sowie Coronabefürworter entspricht den Gruppierungen und wird zur Verständlichkeit dargestellt.

\subsubsection{Komplement $A^{c}$}
%kurze Beschreibung fehlt
$A^{c} \equiv \vec{A}$ := \{$x \in G$ : $x \not\in A$\} = $G \setminus A$ \\
\textbf{Beispiel:} $G$ := $\mathbb{N}$, $A$ := \{0, 1, 2, 3\}, $A^{c} \equiv \vec{A}$ := \{4, 5, 6, 7, ..\}

\subsubsection{Kartesisches Produkt $\times$}
Dies bezeichnet eine Menge geordneter Paare: \\
$A \times B$ := \{$(x, y) | x \in A \land y \in B$\} \\
\textbf{Beispiel:} $A \times B$ := \{$(2,8), (2,16), (4,8), (4,16)$\}

\subsection{Gesetze der Mengenalgebra}
Generell ist zusätzlich nur die Elementeoperation $\in$ notwendig, alle weiteren Operatoren können von der Logik übernommen werden. \\
Der Grund für die Mengenalgebra ist, dass möglichst einfach die Gleichheit von Mengen getestet werden kann.

\subsubsection{Bindungsstärken (Prioritäten)}
\begin{tabular}{l|l|l}
  Bindungsstärke & (Aussagen- und Prädikaten-) Logik & Mengentheorie \\
  \hline
  Höchste & \textit{Klammern} & \textit{Klammern} \\
   & \( \forall, \exists, \exists! \) & \( ^{c}  \phantom{A}(bzw. \overline{\phantom{A}})\) \\
   & \(\lnot\) & \(\bigcup, \bigcap, \Pi\)  \phantom{abcdef}(D. h. $\bigcap\limits_{k=1}^n,\bigcup\limits_{k=1}^n,\prod\limits_{k=1}^{n}$)\\
  & \(\in,\not\in,\land,\lor,\textit{XOR},\subseteq,\subset\,=,\neq\) & \(\cup,\cap,\Delta,\setminus,\times\) \\
  Niedrigste & \(\Longrightarrow,\Longleftrightarrow\) \\
  \hline
\end{tabular}

\subsubsection{Rechenregeln für Mengen}
\begin{tabular}{l|l|l}
  \textbf{Name/Stichwort:} & \textbf{Regel/Gesetz Durchschnitt} & \textbf{Vereinigung}\\
  \hline
  Idempotenzgesetze & \(A \cap A = A \) & \( A \cup A = A \) \\
  \hline
  Kommutativgesetze & \(A \cap B = B \cap A \) & \(A \cup B = B \cup A \) \\
  \hline
  Identitätsgesetze & \(A \cap G = A \) & \(A \cup \emptyset = A \) \\
  & \(A \cap \emptyset = \emptyset\) \\
  \hline
  Assoziativgesetze & \((A \cap B) \cap C = A \cap (B \cap C) \) & \((A \cup B) \cup C = A \cup (B \cup C) \) \\
  \hline
  Absorptionsgesetze & \(A \cap (A \cup B) = A \) & \(A \cup (A \cap B) = A \) \\
  \hline
  Distributivgesetze & \(A \cap (B \cup C) = (A \cap B) \cup (A \cap C) \) & \(A \cup (B \cap C) = (A \cup B) \cap (A \cup C) \) \\
  \hline
  De Morgan Gesetze & \((A \cap B)^{c} = A^{c} \cup B^{c} \) & \((A \cup B)^{c} = A^{c} \cap B^{c} \) \\
  \hline
  Komplementgesetze & \(A \cap A^{c} = \emptyset \) & \(A \cup A^{c} = G \) \\
  & \((A^{c})^{c} = A \) \\
  & \(G^{c} = \emptyset \) \\
  & \(\emptyset^{c} = G \) \\
  \hline
  Teilmengenbeziehungen & \(A \subseteq B \Longrightarrow (A \cap B = A)\) & \(A \subseteq B \Longrightarrow (A \cup B = B) \) \\
  & \((A \subseteq B) \land (B \subseteq C) \Longrightarrow (A \subseteq C) \) \\
  \hline
\end{tabular}

\subsubsection{Partition einer Menge}
Bei einer Partition einer Menge sprechen wir davon, das wir alle Elemente einer Menge in einzelne Gruppen aufteilen, welche schliesslich die so genannten Partitionen bilden.

\textbf{Beispiel:} Wir haben die Menge $\mathbb{N}_8$ (:= \{0,1,2..,8\}). \\
Ein mögliches Mengensystem wäre nun: \\
$S$ := \{\{0,2,4\},\{1,3,7\},\{5,8\},\{6\}\}

Zu erwähnen ist, dass die kombinatorischen Möglichkeiten auf der Bell'schen Zahl basieren. Diese wird schnell extrem gross, so dass bei wenigen Elementen in einer Menge bereits eine riesige Möglichkeit an Kombinationen herrscht.

\subsubsection{Unendlichkeit einer Menge}
Vorerst ist zu sagen, dass es verschiedene Mengen gibt, und zwar \textbf{endliche} oder \textbf{unendliche} Mengen.

Anhand der Kardinalität der Mengen sieht man, ob es jeweils eine Paarung eines Elements jeder Menge gibt. Dadurch lässt sich definieren, ob eine Menge \textbf{unendlich oder endlich} ist. Ausserdem sprechen wir hier von abzählbar unendlich, was soviel heisst wie gewisse Mengen (z.B. Zahlenmengen wie $\mathbb{N}$ oder $\mathbb{Z}$ sind abzählbar unendlich). Wiederum die reellen Zahlen $\mathbb{R}$ sind jedoch nicht abzählbar unendlich (periodische Zahlen/Wurzeln etc.) und nennt sich \textbf{überabzählbar unendlich}.

\textbf{Beispiele und Definition:} \\
$|\mathbb{Z}^{-}| = |\mathbb{Z}^{+}| = |\mathbb{Z}| = |\mathbb{N}_g| = |\mathbb{N}_u| = |\mathbb{Q}| = |\mathbb{N}|$

$\mathbb{R}$ ist überabzählbar unendlich.


\subsubsection*{Beispiele zur Mengenlehre}
$x \in $ M : x ist ein Element von der Menge M \\
\\
\textbf{Aufzählende Form:} \\
A := \{Biene, Wespe, Marienkäfer\} ist eine Menge von Insekten \\
B := \{BMW, Audi, Ferrari\} ist eine Menge von Automarken \\
\\
\textbf{Beschreibende Form} \\
Hier werden alle \textbf{ungeraden natürlichen Zahlen kleiner als 10} in die Menge A aufgenommen. \\
\{\(n : n \in \mathbb{N} \land n < 10 \land (\exists k \in \mathbb{N} : n = 2k -1)\)\} \\

\textbf{Venn Diagramme} \\
\\
Hier dargestellt als Diagramm die Formel der 3 Mengen \(A, B, C\): \((A \cap B) \cup (A \cap C) \cup (B \cap C)\) \\
Im zweiten Diagramm sieht man noch die Vereinfachung davon: \((A \cap C) \)\\
\\
\def\firstcircle{(90:1.25cm) circle (1.5cm)}
  \def\secondcircle{(210:1.25cm) circle (1.5cm)}
  \def\thirdcircle{(330:1.25cm) circle (1.5cm)}
    \begin{tikzpicture}
      \begin{scope}
    \clip \secondcircle;
    \fill[cyan] \thirdcircle;
      \end{scope}
      \begin{scope}
    \clip \firstcircle;
    \fill[cyan] \thirdcircle;
      \end{scope}
      \begin{scope}
    \clip \firstcircle;
    \fill[cyan] \secondcircle;
      \end{scope}
      \draw \firstcircle node[text=black,above] {$A$};
      \draw \secondcircle node [text=black,below left] {$B$};
      \draw \thirdcircle node [text=black,below right] {$C$};
    \end{tikzpicture}
    \begin{tikzpicture}
      \begin{scope}
    \clip \firstcircle;
    \fill[cyan] \thirdcircle;
      \end{scope}
      \draw \firstcircle node[text=black,above] {$A$};
      \draw \secondcircle node [text=black,below left] {$B$};
      \draw \thirdcircle node [text=black,below right] {$C$};
    \end{tikzpicture}
\\
Man sieht schön, wie ein Venn Diagramm aufgebaut ist. Dies gilt natürlich auch für alle anderen Operationen. \\

\textbf{Kartesisches Produkt}
\begin{table}[htb]
\centering
\begin{tabular}{llllll}
                       &                        &       & A     & x                          & B \\ \cline{2-5}
\multicolumn{1}{l|}{B} & \multicolumn{1}{l|}{3} & (a,3) & (b,3) & \multicolumn{1}{l|}{(c,3)} &   \\
\multicolumn{1}{l|}{}  & \multicolumn{1}{l|}{2} & (a,2) & (b,2) & \multicolumn{1}{l|}{(c,2)} &   \\
\multicolumn{1}{l|}{}  & \multicolumn{1}{l|}{1} & (a,1) & (b,1) & \multicolumn{1}{l|}{(c,1)} &   \\ \cline{2-5}
                       & \multicolumn{1}{l|}{}  & a     & b     & \multicolumn{1}{l|}{c}     & A \\ \cline{3-5}
\end{tabular}
\end{table}
\\
Hier sieht man, wie ein kartesisches Produkt aufgebaut ist. Dabei wird \textbf{jedes Element} (der Menge A) \textbf{mit jedem Element} der anderen Menge (B) \textbf{gekreuzt}.
$A^c \equiv \overline{\rm A}$
\end{document}