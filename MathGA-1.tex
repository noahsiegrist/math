\documentclass{article}

\usepackage[utf8]{inputenc} %Umlaute etc.%
\usepackage[ngerman]{babel} %German charset%
\usepackage{color}
\usepackage{amssymb}
\usepackage{amsmath}
\usepackage{amsthm}
\usepackage{datetime}
\usepackage{graphicx}
\usepackage{hyperref}

\setlength\parindent{0pt}

\title{Mathe GA-1}
\author{Jan, Mariel, Noah Siegrist}
\date{\today}

\begin{document}

\maketitle	
\tableofcontents
\newpage

\section{Logik}
Die Logik dient dazu komplexe Probleme analytisch zu Lösen. Sie bricht die Sprache auf eine abstrakte form herunter, die eindeutig und genau ist. Somit ist die Logik ein Grundbaustein aller Wissenschaften.

\subsection{Logische Aussage}

\subsubsection{Definition}

Eine (logische) Aussage kann immer als \textbf{wahr (w)} order \textbf{falsch (f)} identifiziert werden. Wenn die Aussage nicht \textbf{eindeutig} mit w oder f beantwortet werden kann, so ist die Aussage nicht logisch.

\begin{itemize}
	\item[] Eine Aussage kann auch aus mehreren \textbf{Teilaussagen} bestehen. Dann heisst sie \textbf{zusammengesetzte Aussage} oder auch \textbf{aussagenlogische Formel}.
	\item[] Die \textbf{Elementaraussage} oder auch \textbf{atomare Aussage} ist eine Aussage welche aus einer Teilaussage besteht.
\end{itemize}

\subsubsection{Beispiele}

Logische Aussagen:
\begin{itemize}
  \item Feuer gibt wärme ab. (w)
  \item Feuer kühlt. (f)
  \item Feuer ist schön. (subjektive aber bestimmbar.)
  \item Wenn das Feuer brennt, trocknen die Kleider schneller. (w)
\end{itemize}

Keine Logische Aussagen:
\begin{itemize}
  \item Ist das Feuer warm? (Frage)
  \item Feuer kühlt und wärmt. (Widerspruch, Paradoxem)
  \item Mach Feuer! (Befehl)
  \item Das Feuer ist x grad warm. (“Aussageform” (siehe später), aber keine Aussage!)
\end{itemize}

\subsection{Aussagenlogische Verknüspfungen}
Logische Aussagen können verknüpft werden mittels Aussagenlogische Verknüpfungen oder auch “Konnektoren”, “Junktoren” oder “Operatoren” gennant. Diese Verknüpfungen können jeweils mit \textbf{Wahrheitstabellen} definiert werden.

\subsubsection{Negation (nicht)}
Die negation ist keine Verknüpfung sondern invertiert eine Logische Aussage. Sie bindet am stärksten und bildet somit auch ein Literal. 

\textbf{Nicht} A wird wie folgt geschrieben: \( \neg A \)

\subsubsection*{Wahrheitstabelle}
\begin{tabular}{c || c}
  A & \( \neg A \) \\
  \hline
  0  & 1 \\
  1  & 0\\
\end{tabular}\break

\subsubsection*{Beispiele}

\begin{itemize}
  \item Feuer ist \textbf{nicht} kalt. (w)
  \item Feuer ist \textbf{nicht} heiss. (f)
\end{itemize}

\subsubsection{Konjunktion (und)}
A \textbf{und} B wird wie folgt geschrieben: \( A \land B \)

Sie bindet am zweit stärksten direkt nach der Negation.
\subsubsection*{Wahrheitstabelle}
\begin{tabular}{c|c || c}
  A & B & \( A \land B \) \\
  \hline
  0 & 0 & 0 \\
  0 & 1 & 0\\
  1 & 0 & 0\\
  1 & 1 & 1\\
\end{tabular}

\subsubsection*{Beispiele}

\begin{itemize}
  \item Feuer ist kalt \textbf{und} Feuer ist flüssig. (f)
  \item Feuer ist kalt \textbf{und} Feuer ist eine chemische Reaktion. (f)
  \item Feuer ist heiss \textbf{und} Feuer ist flüssig. (f)
  \item Feuer ist heiss \textbf{und} Feuer ist eine chemische Reaktion. (w)
\end{itemize}


\subsubsection{Disjunktion (oder)}
A \textbf{oder} B wird wie folgt geschrieben: \( A \lor B \)

Sie bindet ebenfalls am zweit stärksten direkt nach der Negation.
\subsubsection*{Wahrheitstabelle}
\begin{tabular}{c|c || c}
  A & B & \( A \lor B \) \\
  \hline
  0 & 0 & 0 \\
  0 & 1 & 1\\
  1 & 0 & 1\\
  1 & 1 & 1\\
\end{tabular}\break

Beim Logischem oder ist anzumerken, dass es sich vom entweder oder unterschiedet indem, dass wenn beide Seiten korrekt sind ist das Ergebnis auch Korrekt.

\subsubsection*{Beispiele}

\begin{itemize}
  \item Feuer ist kalt \textbf{oder} Feuer ist flüssig. (f)
  \item Feuer ist kalt \textbf{oder} Feuer ist eine chemische Reaktion. (w)
  \item Feuer ist heiss \textbf{oder} Feuer ist flüssig. (w)
  \item Feuer ist heiss \textbf{oder} Feuer ist eine chemische Reaktion. (w)
\end{itemize}

\subsubsection{Implikation (wenn, dann)}

\textbf{Wenn} A, \textbf{dann} B wird wie folgt geschrieben: \( A \implies B \)

Sie bindet am dritt stärksten nach der Konjunktion und der Disjunktion.
\subsubsection*{Wahrheitstabelle}
\begin{tabular}{c|c || c}
  A & B & \( A \implies B \) \\
  \hline
  0 & 0 & 1 \\
  0 & 1 & 1\\
  1 & 0 & 0\\
  1 & 1 & 1\\
\end{tabular}\break

Wenn A nicht war ist, kann B wahr oder falsch sein, die Aussage ist damit jedoch \textbf{nicht falsch}!

\subsubsection*{Beispiele}
Wenn Max am Feuer sitzt, hat er warm.
\begin{itemize}
  \item Max sitzt nicht am Feuer und er hat kalt. 
  \textit{(Wenn Max nicht am Feuer sitzt, ist es nicht mein Problem wenn er kalt hat. Die Aussage ist nicht falsch, also wahr.)}
  \item Max sitzt nicht am Feuer und er hat warm. 
  \textit{(Wenn Max nicht am Feuer sitzt, kann er trotzdem warm haben. Die Aussage ebenfalls nicht falsch, also wahr.)
}  \item Max sitzt am Feuer und er hat kalt. 
  \textit{(Trotzdem das Max am Feuer sitzt hat er Kalt. Die Aussage ist nun falsch.)}
  \item Max sitzt am Feuer und er hat warm. \textit{(Wahr)}
\end{itemize}
\newpage
\subsubsection{Äquivalenz (genau dann, wenn)}

A \textbf{Genau dann, wenn} B wird wie folgt geschrieben: \( A \iff B \)

Sie bindet ebenfalls am dritt stärksten nach der Konjunktion und der Disjunktion.
\subsubsection*{Wahrheitstabelle}
\begin{tabular}{c|c || c}
  A & B & \( A \iff B \) \\
  \hline
  0 & 0 & 1 \\
  0 & 1 & 0\\
  1 & 0 & 0\\
  1 & 1 & 1\\
\end{tabular}\break

\subsubsection*{Beispiel}
Genau dann, wenn Max am Feuer sitzt, hat er warm.
\begin{itemize}
  \item Max sitzt nicht am Feuer und er hat kalt. 
  \textit{(Wahr)}
  \item Max sitzt nicht am Feuer und er hat warm. 
  \textit{(Max hat auch warm ohne, dass er dazu am Feuer sitzen muss. Aussage falsch.)}
  \item Max sitzt am Feuer und er hat kalt. 
  \textit{(Trotzdem das Max am Feuer sitzt hat er Kalt. Aussage falsch.)}
  \item Max sitzt am Feuer und er hat warm. \textit{(Wahr)}
\end{itemize}
\subsection{Aussagenlogische Formel}
In der Aussagenlogischen Formel können nun komplexe Zusammenhänge erstellt werden. Wie zum Beispiel:

\begin{displaymath}
 f = (A \land B) \implies C
\end{displaymath}

\subsubsection{Syntax}
Regeln wie elementare Zeichen (Symbole) – oder auch Symbolgruppen – korrekt
zusammengesetzt werden dürfen. 

\subsubsection{Semantik}
Bedeutungslehre. Zwei aussagenlogische Formeln sind semantisch gleich oder äquivalent (symbolisch \(f \equiv g\)), wenn ihre Wahrheitstabellen identisch sind. Beachte das Symbol unterscheidet sich vom \( \iff \).

Daher, dass gewisse Ausdrücke Semantisch äquivalent sein können ergeben sich folgende Rechengesetze:

\begin{tabular}{l|cc}
  Name & Gesetz \\
  \hline
  Idempotenzgesetz & \(A \land A \equiv A  \) & \(A \lor A \equiv A  \)\\
  \hline
  Kommutativgesetz & \(A \land B \equiv B \land A \) & \(A \lor B \equiv B \lor A \) \\
  \hline
  Identitatsgesetz & \(A \land true \equiv A \)   & \(A \lor true \equiv true \) \\
  & \(A \land false \equiv false \) & \(A \lor false \equiv A \)\\
  \hline
  Assoziativgesetz & \((A \land B) \land C \equiv A \land (B \land C)\) & \((A \lor B) \lor C \equiv A \lor (B \lor C)\) \\
  \hline
  Absorptionsgesetz & \(A \land (A \lor B) \equiv A\) & \(A \lor (A \land B)\equiv A\) \\
  \hline
  Distributivitatsgesetz & \(A \land (B \lor C) \equiv (A \land B) \lor (A \land C)\) & \(A \lor (B \land C) \equiv (A \lor B) \land (A \lor C)\) \\
  \hline
  De Morgan Gesetz & \(\neg(A \land B)\equiv \neg A \land \neg B\) & \(\neg(A \lor B)\equiv \neg A \lor \neg B\) \\
  \hline
  Negationsgesetz & \(\neg\neg A \equiv A\)\\
\end{tabular}


Wobei sich das Absorptionsgesetz auf dem Idempotenzgesetz und dem Identitatsgesetz aufbaut.

\subsection{Normalformen}
Das Ziel der Normalformen ist es eine Komplexe Formel in eine semantisch äquivalente einfacher zu verstehende form zu bringen.
\subsubsection{Konjunktive Normalform (KNF)}




































\end{document}