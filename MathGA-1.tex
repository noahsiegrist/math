\documentclass[a4paper,12pt]{article}

\usepackage{blindtext} %optimizes spaces & layout stuff, can fill some blindtext so check layout behaviour%
\usepackage{avant} %font%
\usepackage{microtype} %justification of document%
\usepackage{wrapfig} %wrap text around pictures%
\usepackage{enumitem} %optimize list layout, more compact%
\usepackage{fancyhdr} %testing styles%
\usepackage{index} %custom indexing%
\usepackage[singlespacing]{setspace} %set spacing property%
\usepackage{tabularx} %format tabular%

\usepackage[utf8]{inputenc} %Umlaute etc.%
\usepackage[ngerman]{babel} %German charset%
\usepackage{color}
\usepackage{amssymb}
\usepackage{amsmath}
\usepackage{amsthm}
\usepackage{datetime}
\usepackage{graphicx}
\usepackage{hyperref}
\setlength{\parskip}{1em}
\setlength\parindent{0pt}

\usepackage{geometry}
\geometry{a4paper, total={170mm,257mm},left=30mm,right=20mm,top=25mm,bottom=15mm}

\makeindex

\title{\large{\textbf{Mathe GA-1}}}
\author{Mariel Alaba, Jan Meier, Noah Siegrist}
\date{\today}

\usepackage{titling}
\renewcommand\maketitlehooka{\null\mbox{}\vfill}
\renewcommand\maketitlehookd{\vfill\null}

\begin{document}

\begin{titlingpage}
\maketitle
\end{titlingpage}

\newpage
\cfoot[plain]{scrheadings}
\tableofcontents

\newpage

\pagenumbering{arabic}
\setcounter{page}{1}
\pagestyle{fancy}
\fancyhf{} %clear footers style%
\fancyhead[LE,RO]{\nouppercase{\leftmark}}
\fancyfoot[LE,RO]{\centering{\thepage}}

\newpage

\section{Logik}
\enlargethispage{\baselineskip}
Die Logik dient dazu komplexe Probleme analytisch zu Lösen. Sie bricht die Sprache auf eine abstrakte Form herunter, die eindeutig und genau ist. Somit ist die Logik ein Grundbaustein aller Wissenschaften.

\subsection{Logische Aussage}

\subsubsection{Definition}

Eine (logische) Aussage kann immer als \textbf{wahr (w)} order \textbf{falsch (f)} identifiziert werden. Wenn die Aussage nicht \textbf{eindeutig} mit w oder f beantwortet werden kann, so ist die Aussage nicht logisch.

\setlist{nolistsep}
\begin{itemize}
	\item[] Eine Aussage kann auch aus mehreren \textbf{Teilaussagen} bestehen. Dann heisst sie \textbf{zusammengesetzte Aussage} oder auch \textbf{aussagenlogische Formel}.
	\item[] Die \textbf{Elementaraussage} oder auch \textbf{atomare Aussage} ist eine Aussage welche aus einer Teilaussage besteht.
\end{itemize}

\subsubsection{Beispiele}

Logische Aussagen:
\begin{itemize}
  \item Feuer gibt wärme ab. (w)
  \item Feuer kühlt. (f)
  \item Feuer ist schön. (subjektive aber bestimmbar.)
  \item Wenn das Feuer brennt, trocknen die Kleider schneller. (w)
\end{itemize}

Keine Logische Aussagen:
\begin{itemize}
  \item Ist das Feuer warm? (Frage)
  \item Feuer kühlt und wärmt. (Widerspruch, Paradoxem)
  \item Mach Feuer! (Befehl)
  \item Das Feuer ist x grad warm. (“Aussageform” (siehe später), aber keine Aussage!)
\end{itemize}
\newpage
\subsection{Aussagenlogische Verknüpfungen}
Logische Aussagen können verknüpft werden mittels Aussagenlogische Verknüpfungen oder auch “Konnektoren”, “Junktoren” oder “Operatoren” gennant. Diese Verknüpfungen können jeweils mit \textbf{Wahrheitstabellen} definiert werden.

\subsubsection{Negation (nicht)}
Die negation ist keine Verknüpfung sondern invertiert eine Logische Aussage. Sie bindet am stärksten und bildet somit auch ein Literal. 

\textbf{Nicht} A wird wie folgt geschrieben: \( \neg A \)

\subsubsection*{Wahrheitstabelle}
\begin{tabular}{c || c}
  A & \( \neg A \) \\
  \hline
  0  & 1 \\
  1  & 0\\
\end{tabular}\break

\subsubsection*{Beispiele}

\begin{itemize}
  \item Feuer ist \textbf{nicht} kalt. (w)
  \item Feuer ist \textbf{nicht} heiss. (f)
\end{itemize}

\subsubsection{Konjunktion (und)}
A \textbf{und} B wird wie folgt geschrieben: \( A \land B \)

Sie bindet am zweit stärksten direkt nach der Negation.
\subsubsection*{Wahrheitstabelle}
\begin{tabular}{c|c || c}
  A & B & \( A \land B \) \\
  \hline
  0 & 0 & 0 \\
  0 & 1 & 0\\
  1 & 0 & 0\\
  1 & 1 & 1\\
\end{tabular}

\subsubsection*{Beispiele}

\begin{itemize}
  \item Feuer ist kalt \textbf{und} Feuer ist flüssig. (f)
  \item Feuer ist kalt \textbf{und} Feuer ist eine chemische Reaktion. (f)
  \item Feuer ist heiss \textbf{und} Feuer ist flüssig. (f)
  \item Feuer ist heiss \textbf{und} Feuer ist eine chemische Reaktion. (w)
\end{itemize}


\subsubsection{Disjunktion (oder)}
A \textbf{oder} B wird wie folgt geschrieben: \( A \lor B \)

Sie bindet ebenfalls am zweit stärksten direkt nach der Negation.
\subsubsection*{Wahrheitstabelle}
\begin{tabular}{c|c || c}
  A & B & \( A \lor B \) \\
  \hline
  0 & 0 & 0 \\
  0 & 1 & 1\\
  1 & 0 & 1\\
  1 & 1 & 1\\
\end{tabular}\break

Beim Logischem oder ist anzumerken, dass es sich vom entweder oder unterschiedet indem, dass wenn beide Seiten korrekt sind ist das Ergebnis auch Korrekt.

\subsubsection*{Beispiele}

\begin{itemize}
  \item Feuer ist kalt \textbf{oder} Feuer ist flüssig. (f)
  \item Feuer ist kalt \textbf{oder} Feuer ist eine chemische Reaktion. (w)
  \item Feuer ist heiss \textbf{oder} Feuer ist flüssig. (w)
  \item Feuer ist heiss \textbf{oder} Feuer ist eine chemische Reaktion. (w)
\end{itemize}

\subsubsection{Implikation (wenn, dann)}

\textbf{Wenn} A, \textbf{dann} B wird wie folgt geschrieben: \( A \implies B \)

Sie bindet am dritt stärksten nach der Konjunktion und der Disjunktion.
\subsubsection*{Wahrheitstabelle}
\begin{tabular}{c|c || c}
  A & B & \( A \implies B \) \\
  \hline
  0 & 0 & 1 \\
  0 & 1 & 1\\
  1 & 0 & 0\\
  1 & 1 & 1\\
\end{tabular}\break

Wenn A nicht war ist, kann B wahr oder falsch sein, die Aussage ist damit jedoch \textbf{nicht falsch}!

\pagebreak

\subsubsection*{Beispiele}
Wenn Max am Feuer sitzt, hat er warm.
\begin{itemize}
  \item Max sitzt nicht am Feuer und er hat kalt. 
  \textit{(Wenn Max nicht am Feuer sitzt, ist es nicht mein Problem wenn er kalt hat. Die Aussage ist nicht falsch, also wahr.)}
  \item Max sitzt nicht am Feuer und er hat warm. 
  \textit{(Wenn Max nicht am Feuer sitzt, kann er trotzdem warm haben. Die Aussage ebenfalls nicht falsch, also wahr.)
}  \item Max sitzt am Feuer und er hat kalt. 
  \textit{(Trotzdem das Max am Feuer sitzt hat er Kalt. Die Aussage ist nun falsch.)}
  \item Max sitzt am Feuer und er hat warm. \textit{(Wahr)}
\end{itemize}

\subsubsection{Äquivalenz (genau dann, wenn)}

A \textbf{Genau dann, wenn} B wird wie folgt geschrieben: \( A \iff B \)

Sie bindet ebenfalls am dritt stärksten nach der Konjunktion und der Disjunktion.
\subsubsection*{Wahrheitstabelle}
\begin{tabular}{c|c || c}
  A & B & \( A \iff B \) \\
  \hline
  0 & 0 & 1 \\
  0 & 1 & 0\\
  1 & 0 & 0\\
  1 & 1 & 1\\
\end{tabular}\break

\subsubsection*{Beispiel}
Genau dann, wenn Max am Feuer sitzt, hat er warm.
\begin{itemize}
  \item Max sitzt nicht am Feuer und er hat kalt. 
  \textit{(Wahr)}
  \item Max sitzt nicht am Feuer und er hat warm. 
  \textit{(Max hat auch warm ohne, dass er dazu am Feuer sitzen muss. Aussage falsch.)}
  \item Max sitzt am Feuer und er hat kalt. 
  \textit{(Trotzdem das Max am Feuer sitzt hat er Kalt. Aussage falsch.)}
  \item Max sitzt am Feuer und er hat warm. \textit{(Wahr)}
\end{itemize}
\subsection{Atomare Aussage}
Eine Atomare Aussage, ist eine Aussage ohne jeglichen Verknüpfungen oder Negationen. Also bei der Aussagelogischer Formel ( \(f = A \land \neg B \land B\) ) sind die Atomaren Aussagen ( \(A\) und \(B\) ).

\subsection{Literal}
Ein Literal, ist eine Aussage ohne jeglichen Verknüpfungen. Im Gegensatz zur Atomaren Aussage sind hier Negationen nicht ausgeschlossen. Also bei der Aussagelogischer Formel ( \(f = A \land \neg B \land B\) ) sind die Literal ( \(A\), \(B\) und \(\neg B\) ).



%\newpage
\subsection{Aussagenlogische Formel}
In der Aussagenlogischen Formel können nun komplexe Zusammenhänge erstellt werden. Wie zum Beispiel:

\begin{displaymath}
 f = (A \land B) \implies C
\end{displaymath}

\subsubsection{Syntax}
Regeln wie elementare Zeichen (Symbole) – oder auch Symbolgruppen – korrekt
zusammengesetzt werden dürfen. 

\subsubsection{Semantik}
Bedeutungslehre. Zwei aussagenlogische Formeln sind semantisch gleich oder äquivalent (symbolisch \(f \equiv g\)), wenn ihre Wahrheitstabellen identisch sind. Beachte das Symbol unterscheidet sich vom \( \iff \).

Daher, dass gewisse Ausdrücke Semantisch äquivalent sein können ergeben sich folgende Rechengesetze:

\begin{tabular}{l|cc}
  Name & Gesetz \\
  \hline
  Idempotenzgesetz & \(A \land A \equiv A  \) & \(A \lor A \equiv A  \)\\
  \hline
  Kommutativgesetz & \(A \land B \equiv B \land A \) & \(A \lor B \equiv B \lor A \) \\
  \hline
  Identitatsgesetz & \(A \land true \equiv A \)   & \(A \lor true \equiv true \) \\
  & \(A \land false \equiv false \) & \(A \lor false \equiv A \)\\
  \hline
  Assoziativgesetz & \((A \land B) \land C \equiv A \land (B \land C)\) & \((A \lor B) \lor C \equiv A \lor (B \lor C)\) \\
  \hline
  Absorptionsgesetz & \(A \land (A \lor B) \equiv A\) & \(A \lor (A \land B)\equiv A\) \\
  \hline
  Distributivitatsgesetz & \(A \land (B \lor C) \equiv (A \land B) \lor (A \land C)\) & \(A \lor (B \land C) \equiv (A \lor B) \land (A \lor C)\) \\
  \hline
  De Morgan Gesetz & \(\neg(A \land B)\equiv \neg A \land \neg B\) & \(\neg(A \lor B)\equiv \neg A \lor \neg B\) \\
  \hline
  Negationsgesetz & \(\neg\neg A \equiv A\)\\
\end{tabular}


Wobei sich das Absorptionsgesetz auf dem Idempotenzgesetz und dem Identitatsgesetz aufbaut.

\subsection{Normalformen}
Das Ziel der Normalformen ist es eine Komplexe Formel in eine semantisch äquivalente einfacher zu verstehende form zu bringen.
\subsubsection{Konjunktive Normalform (KNF)}
In der KNF ist es das Ziel, dass im obersten level der Aussagelogischen Formel nur Konjunktionen vorhanden sind. Diese halten so genante "Klauseln der konjunktiven Normalform" zusammen.
\begin{displaymath}
  f = f_1 \land f_2 \land ... \land f_n
\end{displaymath}

Die \(f_1\) bis \(f_n\) sind dabei die Klauseln, wobei eine solche Klausel aus Disjunktion und Literalen bestehen. Hier am Beispiel von der ersten Klausel:
\begin{displaymath}
  f_1 = L_{11} \lor L_{12} \lor ... \lor L_{1n}
\end{displaymath}

\subsubsection{Disjunktive Normalform (KNF)}

In der DNF ist es das Ziel, dass im obersten level der Aussagelogischen Formel nur Disjunktionen vorhanden sind. Diese halten so genante "Klauseln der disjunktiven Normalform" zusammen.
\begin{displaymath}
  f = f_1 \lor f_2 \lor ... \lor f_n
\end{displaymath}

Die \(f_1\) bis \(f_n\) sind dabei die Klauseln, wobei eine solche Klausel aus Disjunktion und Literalen bestehen. Hier am beispiel von der ersten Klausel:
\begin{displaymath}
  f_1 = L_{11} \land L_{12} \land ... \land L_{1n} 
\end{displaymath}


\subsubsection{Beispiele}
\begin{itemize}
  \item \((A \lor B) \land (C \lor \neg D) \land (\neg E \lor F)  \) (KNF)
  \item \((A \land B) \lor (C \land \neg D) \lor (\neg E \land F)  \) (DNF)
  \item \((A \lor B) \lor (C \land \neg D) \lor (\neg E \land F)  \) (Diese Formel kann weiter zur DNF gebracht werden indem die ersten Klammern entfernt werden dank dem Assoziativgesetz.)
  	
  	 \(A \lor B \lor (C \land \neg D) \lor (\neg E \land F)  \)
  \item \((A \lor \neg (C \land D)) \land (\neg E \lor F)  \) (Diese Formel kann weiter zur KNF gebracht werden mit dem dem de Morgan Gesetz und dem Distributivitatsgesetz.)
 	\((A \lor (\neg C \land \neg D)) \land (\neg E \lor F)  \)
 	
  	\( (A \lor \neg C) \land (A \lor \neg D) \land (\neg E \lor F)  \)
    
\end{itemize}

\subsection{Belegung und Modelle}
Unter einer Belegung verstehen wir eine Zuordnung der Atomaren Aussagen zu einem effektiven Wert. So zu sagen eine Zeile aus der Wahrheitstabelle anders repräsentiert.

Wenn diese Belegung nun die Aussagelogische Formel wahr macht, so wird die Belegung Modell genannt.

%TODO: Beispiel für Belegung und Model... Wie wird das geschrieben?

\subsection{Erfüllbarkeit}
Wenn eine Aussagenlogische Formel wird \textbf{erfüllbar} gennant, wenn mindestens ein Modell in der Wahrheitstabelle gefunden werden kann. Falls kein Modell gefunden werden kann, so handelt es sich um eine \textbf{Kontradiktion / Widerspruch}. \textit{Das warme Feuer ist kalt.}

Falls in der Wahrheitstabelle nur Modelle gefunden werden können, so nennen wir die Aussagenlogische Formel eine \textbf{Tautologie}. \textit{Das warme oder kalter Feuer.} 

\subsection{Prädikate}
In der Pradikatenlogik wollen wir jetzt nicht mehr Eigenschaften einzelner Objekte betrachten wie z.B. dem Feuer,
sondern die Eigenschaften einer ganzen Menge (mehr zu Mengen im nächsten Kapitel) an Objekten und machen darüber wieder Aussagen. So sind wir in der Lage effiziente Aussagen zu mehreren Objekten zu machen.

Wir haben eine Menge an Geometrischen formen und wollen herausfinden welche davon rund sind, können wir ein Prädikat aufstellen: \( P(x) := \) "x ist rund".


Unser Prädikat \( P(x) \) wird erst zur Aussage, wenn für x ein Objekt eingesetzt wird. Solange kein Objekt eingesetzt ist, kann das Prädikat auch nicht auf wahr oder falsch definiert werden und ist deshalb auch noch keine Aussage. Wenn nun ein Recheck für x eingesetzt wird, können wir feststellen, dass die Aussage \textit{"Rechteck ist rund"} falsch ist.

Ein Prädikat kann auch mehre Variablen enthalten, oder auch keine, was dann eine Aussage währe ().


\subsubsection*{Beispiel}











\end{document}